\section{Conclusions and future work}
    In conclusion, our solution aims to ease the use of public transport by VIPs. The first step that we have taken in doing this is creating an object detection system that can recognize buses, cars, or vehicle registration plates. This part is implemented using a custom version of YOLOv2 \cite{yolov2} and we obtain, on the test set, a mAP of 70.03\%, and for the bus class, we obtain an average precision of 90.01\%, for the car class 64.04\% and for the vehicle registration plate 56.68\%, with a speed of around 5 FPS on a mobile device. We have also trained a model on the COCO dataset that achieves around 0.4\% mAP on the test dataset. The second part is represented by the mobile application, which serves both as an object detection system visualizer and as a proof of concept for assistive technology for the VIPs that uses object detection. This is illustrated by the accessible live object detection, in which the predictions are not visualized, but converted to sound and played using the mobile device speakers. 
    
    %These results need more work until they reach state of the art, which is real-time speeds of 30-60 FPS and better performance in terms of accuracy.
    
    There are several parts that can be improved, such that the proposed method attains state of the art results both in terms of accuracy and computational complexity, and we leave them as future work. Firstly, the dataset could be enhanced with images with bad lights or weather, or night images. Recent advances have shown that data-centric AI yields better results than model-centric AI, therefore the dataset could use more attention, in the sense that bad ground truth annotations should be found and fixed. Other techniques presented in the other YOLO papers such as training with images of different sizes, could prove useful. 
    
    %On the Android application side, the conversion from bounding box to sound could be improved. For example, we don't use the positions of the bounding boxes, therefore valuable information is not used. The solution would be to develop a sound convention, that could convert the bounding boxes to a compact and easy-to-understand form.
