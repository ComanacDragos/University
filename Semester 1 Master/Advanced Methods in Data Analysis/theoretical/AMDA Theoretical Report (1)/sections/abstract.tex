%Context

Computer vision has emerged as a key field of artificial intelligence because it aims to replicate functions of the human visual cortex. More specifically, computer vision aims to extract information from visual data such as images or videos. For example, object detection is one of the main computer vision problems in which instances of objects must be located and classified in  visual data.

%Objective
The main purpose of this paper is to discuss the arhitectural details of You Only Look Once method for solving object detection. This method belongs to the class of single shot object detectors, meaning that it is most suitable for applications in which speed is crucial, as opposed to its slower two-stage class of object detectors. The speed comes from the fact that the method proposes a special convolutional neural network architecture which is able to solve end to end the object detection problem. We also aim to compare You Only Look Once with other methods for performing object detection.

Over time, this method has proven to be a robust way of performing object detection, given recent applications that use it and achieve competitive results on the standard Microsoft Common Objects in Context dataset for comparing object detectors.
