\section{Placement in the broader field} \label{placement}

Computer vision is a scientific field that deals with the extraction of meaningful information from visual data such as images or videos. As a consequence, over time, computer vision has emerged as a key field in the domain of artificial intelligence because it intends to replicate the functions of the human visual cortex. This is possible due to the continuous development of optical hardware, which nowadays can exceed the capabilities of the human eye, but probably more important are the huge quantities of data that are available to more and more people.

There are various methods that can be used in the field of computer vision, such as hand crafted features, but in this context of big data, the most relevant methods are related to machine learning, and more recently to deep learning which is better suited to visual data. Through supervised learning, deeper and deeper neural network models are now able to learn the complex patterns found in large amounts of data, thus they are a good fit for solving computer vision problems.

One such problem is object detection, which consists of locating and classifying objects in an image. This problem is relevant in many domains such as automotive, logistics or even assistive technologies. Also, it has more potential than simple classification, because the objects are localized and this forces the learning algorithm to look for the very specific patterns that describe the objects, as opposed to classification where other patterns might be learned if the data distribution is unbalanced.

Broadly speaking, in the context of neural networks, the object detection problem mainly branches out in two-stage object detection which traditionally was slow, but accurate and one-stage object detection which initially was less accurate and fast, but recently they became better and better, even surpassing the performance of two-stage object detectors.

The first modern object detectors split the object detection problem into several tasks, which composed a pipeline that is often difficult to train, hence they are collectively named two-stage object detectors. The first steps taken into this direction were made by region-based object detectors such as R-CNN \cite{rcnn} and the later faster versions. Essentially, the bounding boxes are generated through selective search, after which a convolutional neural network is used to extract feature maps that are then classified.

One-stage object detectors achieve real-time speed with good accuracy because their detection pipeline consists only of one convolutional neural network that processes the image and directly outputs the predictions. This approach used to have low accuracy due to the lack of large amounts of data, but recent advances have made the one-stage detectors rival the two-stage detectors in terms of accuracy, without losing speed. YOLO falls into this one-stage class of object detectors, hence it's success in achieving good performance, with little resources.